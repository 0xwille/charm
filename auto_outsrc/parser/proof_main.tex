%%%%%%%%%%%%%%%%%%%%%%%%%%%%%%%%%%%%%%%%%%
% STATIC LATEX CONTENT FOR MACHINE-GENERATED PROOFS
%
% THIS FILE IS NOT GENERATED. IT INCLUDES THE FILE "proof_schemename.tex"
% WHICH CONTAINS THE MACROS FOR THE GUTS OF THE PROOF.
%
% YOU SHOULDN'T NEED TO EDIT THIS FILE EXCEPT FOR STYLISTIC
% REASONS, I.E., YOU DON'T NEED TO CUSTOMIZE IT FOR A GIVEN SCHEME
%%%%%%%%%%%%%%%%%%%%%%%%%%%%%%%%%%%%%%%%%%

\documentclass[11pt]{article}
\usepackage{fullpage,amsthm,amsmath}
\usepackage{latexsym,amssymb,xspace}

\begin{document}
% input for automatically generated outsource proof


\catcode`\^ = 13 \def^#1{\sp{#1}{}}
\newcommand{\newln}{\\&\quad\quad{}}
\newcommand{\schemename}{{\sf Waters09}}
\newcommand{\schemeref}{Waters09_proof}
\newcommand{\schemecite}{\cite{REF}}
\newcommand{\secretkey}{ K,Kl,L }
\newcommand{\listofbfs}{ {\sf bf_0} }
\newcommand{\listofmskvalues}{ \alpha }
\newcommand{\listofrandomvalues}{ t }
\newcommand{\keydefinitions}{ \alpha' = {\alpha \cdot {\sf bf_0}},t' = {t \cdot {\sf bf_0}} }
\newcommand{\originalkey}{ K = g_2^{\alpha} \cdot g_2^{a \cdot t},Kl = (H(S_y))^{t},L = g_2^{t} }
\newcommand{\transformkey}{ K' = g_2^{\alpha'} \cdot g_2^{a \cdot t'},Kl' = (H(S_y))^{t'},L' = g_2^{t'} }
\newcommand{\pseudokey}{ K'',Kl'',L'' }


\newtheorem{definition}{Definition}
\newtheorem{theorem}{Theorem}
\newcommand{\Oracle}{\mathcal{O}}
\newcommand{\Adv}{\mathcal{A}}
\newcommand{\Bdv}{\mathcal{B}}
\newcommand{\MS}{\mathcal{M}}
\newcommand{\Psetup}{\mathsf{PSetup}}
\newcommand{\Msetup}{\mathsf{Setup}}
\newcommand{\params}{\mathit{params}}
\newcommand{\brk}[1]{\langle #1 \rangle}
\newcommand{\ait}[1]{#1}
\newcommand{\Ga}{\ait{\mathbb{G}}_1}
\newcommand{\ga}{\ait{g}_1}
\newcommand{\ha}{\ait{h}_1}
\newcommand{\poly}{\mathrm{poly}}

\newcommand{\bit}[1]{#1}
\newcommand{\Gb}{\bit{\mathbb{G}}_2}
\newcommand{\gb}{\bit{g}_2}
\newcommand{\hb}{\bit{h}_2}

\newcommand{\cit}[1]{#1}
\newcommand{\Gc}{\cit{\Group_T}}
\newcommand{\gc}{\cit{g}}
\newcommand{\hc}{\cit{h}}
\newcommand{\Zp}{\mathbb{Z}_p}

\newcommand{\Group}{\ensuremath{\mathbb{G}}\xspace}
\newcommand{\Hroup}{\ensuremath{\mathbb{H}}\xspace}
\newcommand{\map}{\mathbf{e}}

\newcommand{\prot}{\mathsf{Prot}}
\newcommand{\auxext}{\mathit{auxext}}
\newcommand{\auxsim}{\mathit{auxsim}}
\newcommand{\aux}{\mathit{aux}}
\newcommand{\state}{\mathit{state}}
\newcommand{\Alg}{\mathsf{Alg}}
\newcommand{\A}{\mathcal{A}}

\newcommand{\Sig}{\mathsf{Sig}}
\newcommand{\G}{\mathsf{Gen}}
\newcommand{\SK}{\mathsf{SK}}
\newcommand{\Screen}{\mathsf{Screen}}
\newcommand{\Setup}{\mathsf{Setup}}
\newcommand{\Keygen}{\mathsf{Keygen}}
\newcommand{\KeygenOut}{\mathsf{KeygenOut}}
\newcommand{\Transform}{\mathsf{Transform}}
\newcommand{\Decrypt}{\mathsf{Decrypt}}
\newcommand{\compareequals}{\stackrel{?}{=}}
\newcommand{\numsigs}{\eta}

\title{A machine-generated proof of security for {\schemename}}
\author{}
\date{}
\maketitle

\section{$\KeygenOut$ Proof}

Let $\listofmskvalues$ be the MSK variable(s) and $\listofrandomvalues$ be randomness selected in the $\Keygen$ algorithm. The $\Keygen$ algorithm runs to obtain the secret key, $\SK$ and is computed as follows:

\begin{description}
\item {\sf SK}: \begin{equation}  \originalkey \end{equation}
\end{description}

\noindent
The $\KeygenOut$ algorithm selects blinding factors, $\listofbfs \in \Zp^*$. Let $\keydefinitions$ be the new MSK variables and random values selected in $\KeygenOut$ and let the new $\SK'$ be computed as follows:

\begin{description}
\item {\sf SK'}: \begin{equation}  \transformkey \end{equation}
\end{description}

\noindent
The new $\SK'$ has the same uniform distribution as a randomly generated pseudo-key ($\pseudokey$) without knowledge of the original $\Keygen$. Therefore, the $\SK'$ provides no benefit to an adversary.

\section{$\Transform$ Proof}

\end{document}
