%%%%%%%%%%%%%%%%%%%%%%%%%%%%%%%%%%%%%%%%%%
% STATIC LATEX CONTENT FOR MACHINE-GENERATED PROOFS
%
% THIS FILE IS NOT GENERATED. IT INCLUDES THE FILE "proof_schemename.tex"
% WHICH CONTAINS THE MACROS FOR THE GUTS OF THE PROOF.
%
% YOU SHOULDN'T NEED TO EDIT THIS FILE EXCEPT FOR STYLISTIC
% REASONS, I.E., YOU DON'T NEED TO CUSTOMIZE IT FOR A GIVEN SCHEME
%%%%%%%%%%%%%%%%%%%%%%%%%%%%%%%%%%%%%%%%%%

\documentclass[11pt]{article}
\usepackage{fullpage,amsthm,amsmath}
\usepackage{latexsym,amssymb,xspace}

\begin{document}
% input for automatically generated outsource proof
%

\catcode`\^ = 13 \def^#1{\sp{#1}{}}
\newcommand{\newln}{\\&\quad\quad{}}
\newcommand{\schemename}{{\sf Waters09}}
\newcommand{\schemeref}{Waters09_proof}
\newcommand{\schemecite}{\cite{REF}}
\newcommand{\secretkey}{ K,Kl,L }
\newcommand{\listofbfs}{ {\sf bf_0} }
\newcommand{\listofmskvalues}{ \alpha }
\newcommand{\listofrandomvalues}{ t }
\newcommand{\keydefinitions}{ \alpha' = {\alpha \cdot {\sf bf_0}},t' = {t \cdot {\sf bf_0}} }
\newcommand{\originalkey}{ K = g_2^{\alpha} \cdot g_2^{a \cdot t},Kl = (H(S_y))^{t},L = g_2^{t} }
\newcommand{\transformkey}{ K' = g_2^{\alpha'} \cdot g_2^{a \cdot t'},Kl' = (H(S_y))^{t'},L' = g_2^{t'} }
\newcommand{\pseudokey}{ K'',Kl'',L'' }



\catcode`\^ = 13 \def^#1{\sp{#1}{}}
\newcommand{\newln}{\\&\quad\quad{}}
\newcommand{\schemename}{{\sf Dsewaters09}}
\newcommand{\schemeref}{Dsewaters09_proof}
\newcommand{\schemecite}{\cite{REF}}
\newcommand{\secretkey}{ D_1,D_2,D_3,D_4,D_5,D_6,D_7,K,tag_k }
\newcommand{\listofbfs}{ {\sf uf_0},{\sf bf_0},{\sf uf_1} }
\newcommand{\listofmskvalues}{ \alpha,a_1 }
\newcommand{\listofrandomvalues}{ r_1,r_2,z_2,z_1,id,tag_k }
\newcommand{\keydefinitions}{ D_1' = (D_1 ) ^ { {\sf uf_0}},\alpha' = {\alpha \cdot {\sf bf_0}},z_1' = {z_1 \cdot {\sf bf_0}},r_2' = {r_2 \cdot {\sf bf_0}},r_1' = {r_1 \cdot {\sf bf_0}},z_2' = {z_2 \cdot {\sf bf_0}},K' = (K ) ^ { {\sf uf_1}},tag_k' = {tag_k \cdot {\sf bf_0}} }
\newcommand{\originalkey}{ D_1 = g^{\alpha \cdot a_1} \cdot v^{(r_1 + r_2)},D_2 = g^{-\alpha} \cdot v_1^{(r_1 + r_2)} \cdot g^{z_1},D_3 = g^{b \cdot -z_1},D_4 = v_2^{(r_1 + r_2)} \cdot g^{z_2},D_5 = g^{b \cdot -z_2},\\D_6 = g^{b \cdot r_2},D_7 = g^{r_1},K = (u^{id} \cdot w^{tag_k} \cdot h)^{r_1},tag_k = tag_k }
\newcommand{\transformkey}{ D_1' = (g^{\alpha \cdot a_1} \cdot v^{(r_1 + r_2)})^{{{\sf uf_0}}},D_2' = g^{-\alpha'} \cdot v_1^{(r_1' + r_2')} \cdot g^{z_1'},D_3' = g^{b \cdot -z_1'},D_4' = v_2^{(r_1' + r_2')} \cdot g^{z_2'},D_5' = g^{b \cdot -z_2'},\\D_6' = g^{b \cdot r_2'},D_7' = g^{r_1'},K' = u^{id} \cdot w^{tag_k} \cdot h^{r_1 \cdot {{\sf uf_1}}},tag_k' = tag_k' }
\newcommand{\pseudokey}{ D_1'',D_2'',D_3'',D_4'',\\D_5'',D_6'',D_7'',K'',tag_k'' }

% transform commands
\newcommand{\ciphertext}{ C_1,C_2,C_3,C_4,C_5,C_6,C_7,E_1,E_2,tag_c }

% showing the decrypt routine
\newcommand{\gutsofdecrypt}{
\medskip \noindent
{\bf  Step 1:} Compute tag:
\begin{equation}
tag = (tag_c - tag_k)^{-1}
\end{equation}
\medskip \noindent
{\bf  Step 2:} Compute $A_1$:
\begin{equation}
A_1 = e(C_1, D_1) \cdot e(C_2, D_2) \cdot e(C_3, D_3) \cdot e(C_4, D_4) \cdot e(C_5, D_5)
\end{equation}
\medskip \noindent
{\bf  Step 3:} Compute $A_2$:
\begin{equation}
A_2 = e(C_6, D_6) \cdot e(C_7, D_7)
\end{equation}
\medskip \noindent
{\bf  Step 4:} Compute $A_3$:
\begin{equation}
A_3 = A_1 / A_2
\end{equation}
\medskip \noindent
{\bf  Step 5:} Compute $A_4$:
\begin{equation}
A_4 = e(E_1, D_7) / e(E_2, K)
\end{equation}
\medskip \noindent
{\bf  Step 6:} Compute $R_0$:
\begin{equation}
R_0 = A_4 ^{tag}
\end{equation}
\medskip \noindent
{\bf  Step 7:} Compute $R_1$:
\begin{equation}
R_1 = A_3 / R_0
\end{equation}
\medskip \noindent
{\bf  Step 8:} Compute M:
\begin{equation}
M = C_0 / R_1
\end{equation}
}

\newcommand{\gutsoftransform}{
\medskip \noindent
{\bf  Step 1:} Compute tag:
\begin{equation}
tag = (tag_c - tag_k)^{-1}
\end{equation}
\medskip \noindent
{\bf  Step 2:} Compute $A_1$:
\begin{equation}
BT_1 = e(C_1, D_1'), BT_2 = e(C_2, D_2'), BT_3 = e(C_3, D_3') \cdot e(C_5, D_5'), BT_4 = e(C_4, D_4')
\end{equation}
\medskip \noindent
{\bf  Step 3:} Compute $A_2$:
\begin{equation}
A_2 = e(C_6, D_6') \cdot e(C_7, D_7')
\end{equation}
\medskip \noindent
{\bf  Step 4:} Compute $A_3$:
\begin{equation}
A_3 = A_1 / A_2
\end{equation}
\medskip \noindent
{\bf  Step 5:} Compute $A_4$:
\begin{equation}
A_4 = e(E_1, D_7') / e(E_2, K')
\end{equation}
\medskip \noindent
{\bf  Step 6:} Compute $R_0$:
\begin{equation}
R_0 = A_4 ^{tag}
\end{equation}
%\medskip \noindent
%{\bf  Step 7:} Compute $R_1$:
%\begin{equation}
%R_1 = A_3 / R_0
%\end{equation}
%\medskip \noindent
%{\bf  Step 8:} Compute M:
%\begin{equation}
%M = C_0 / R_1
%\end{equation}
}



\newtheorem{definition}{Definition}
\newtheorem{theorem}{Theorem}
\newcommand{\Oracle}{\mathcal{O}}
\newcommand{\Adv}{\mathcal{A}}
\newcommand{\Bdv}{\mathcal{B}}
\newcommand{\MS}{\mathcal{M}}
\newcommand{\Psetup}{\mathsf{PSetup}}
\newcommand{\Msetup}{\mathsf{Setup}}
\newcommand{\params}{\mathit{params}}
\newcommand{\brk}[1]{\langle #1 \rangle}
\newcommand{\ait}[1]{#1}
\newcommand{\Ga}{\ait{\mathbb{G}}_1}
\newcommand{\ga}{\ait{g}_1}
\newcommand{\ha}{\ait{h}_1}
\newcommand{\poly}{\mathrm{poly}}

\newcommand{\bit}[1]{#1}
\newcommand{\Gb}{\bit{\mathbb{G}}_2}
\newcommand{\gb}{\bit{g}_2}
\newcommand{\hb}{\bit{h}_2}

\newcommand{\cit}[1]{#1}
\newcommand{\Gc}{\cit{\Group_T}}
\newcommand{\gc}{\cit{g}}
\newcommand{\hc}{\cit{h}}
\newcommand{\Zp}{\mathbb{Z}_p}

\newcommand{\Group}{\ensuremath{\mathbb{G}}\xspace}
\newcommand{\Hroup}{\ensuremath{\mathbb{H}}\xspace}
\newcommand{\map}{\mathbf{e}}

\newcommand{\prot}{\mathsf{Prot}}
\newcommand{\auxext}{\mathit{auxext}}
\newcommand{\auxsim}{\mathit{auxsim}}
\newcommand{\aux}{\mathit{aux}}
\newcommand{\state}{\mathit{state}}
\newcommand{\Alg}{\mathsf{Alg}}
\newcommand{\A}{\mathcal{A}}

\newcommand{\Sig}{\mathsf{Sig}}
\newcommand{\G}{\mathsf{Gen}}
\newcommand{\SK}{\mathsf{SK}}
\newcommand{\CT}{\mathsf{CT}}
\newcommand{\Screen}{\mathsf{Screen}}
\newcommand{\Setup}{\mathsf{Setup}}
\newcommand{\Keygen}{\mathsf{Keygen}}
\newcommand{\KeygenOut}{\mathsf{KeygenOut}}
\newcommand{\Transform}{\mathsf{Transform}}
\newcommand{\Decrypt}{\mathsf{Decrypt}}
\newcommand{\Decout}{\mathsf{DecOut}}
\newcommand{\compareequals}{\stackrel{?}{=}}
\newcommand{\numsigs}{\eta}

\title{A machine-generated proof of security for {\schemename}}
\author{}
\date{}
\maketitle

\section{$\KeygenOut$ Proof}

Let $\listofmskvalues$ be the MSK variable(s) and $\listofrandomvalues$ be randomness selected in the $\Keygen$ algorithm. The $\Keygen$ algorithm runs to obtain the secret key, \\ $\SK = \{\secretkey\}$ and is computed as follows:

\begin{description}
\item {\sf SK}: \begin{multline*}  \originalkey \end{multline*}
\end{description}

\noindent
The $\KeygenOut$ algorithm selects blinding factors, $\listofbfs \in \Zp^*$. Let $\keydefinitions$ be the new MSK variables and random values selected in $\KeygenOut$ and let the new $\SK'$ be computed as follows:

\begin{description}
\item {\sf SK'}: \begin{multline*}  \transformkey \end{multline*}
\end{description}

\noindent
The new $\SK'$ has the same uniform distribution as a randomly generated pseudo-key ($\pseudokey$) without knowledge of the original $\Keygen$. Therefore, the $\SK'$ provides no benefit to an adversary.

\section{$\Transform$ Proof}

In the ${\schemename}$ scheme, we show the relevant portions to our transformation in the $\Decrypt$ algorithm. The ciphertext, $\CT = \{\ciphertext\}$ and the $\SK = \{\secretkey\}$. 

%\begin{description}
%\item {Step 1: Compute tag} \begin{equation*} tag =  \end{equation*}
%\end{description}
\gutsofdecrypt

% Rework this text...
Then, we show how the computation of the original $\Decrypt$ algorithm is distributed between the $\Transform$ algorithm and the $\Decout$ algorithm.

\gutsoftransform

\end{document}
